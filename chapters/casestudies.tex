\section{Case Studies}
\label{sec:casestudies}


With support of the composition operators, Reo can be used to capture various coordination scenarios in the real world. In this section, we present two examples: \emph{Probabilistic Router} and \emph{Expiring Timer}.


\begin{example}[Probabilistic Router]
\emph{Router} is a widely used connector example \cite{Baier2006a,Arbab2006}. As shown in Fig.~\ref{fig:router} (a), a \emph{Router} uses two \emph{LossySync} channels and a \emph{SyncDrain} channel to make sure that a coming data value is only sent to one of its sink ends. This choice is made nondeterministically at $C$, where the \emph{Merger} channel exists. Here we show how the nondeterministic behavior is resolved as probabilistic behavior through the \emph{StochasticChoice} channel.


\begin{figure}[t]
    \centering
    \begin{tikzpicture}
    % part 1. router
    \node (note) at (4,2.5) {\small
        ($dist_0$) follows a binomial distribution $\operatorname{B}(1,0.5)$
    };

    \node (note1) at (1.6, -1.4) {\small (a) Router};
    \node (note2) at (6.5, -1.4) {\small (b) Probabilistic Router};

    \ionode{(in)}{(0, 0)}{node[left] {A}}
    \ionode{(outup)}{(3, 0.6)}{node[right] {E}}
    \ionode{(outdown)}{(3, -0.6)}{node[right] {F}}
    \mixednode{(mid)}{(1.5, 0)}{node[right] {C}}
    \mixednode{(wayup)}{(1.5, 0.6)}{node[above] {B}}
    \mixednode{(waydown)}{(1.5, -0.6)}{node[below] {D}}

    \lossysync{(in)}{(wayup)}{}
    \lossysync{(in)}{(waydown)}{}
    \syncdrain{(in)}{(mid)}{}
    \sync{(wayup)}{(mid)}{}
    \sync{(waydown)}{(mid)}{}

    \sync{(wayup)}{(outup)}{}
    \sync{(waydown)}{(outdown)}{}

    \ionode{(pin)}{(5, 0)}{node[left] {A}}
    \ionode{(poutup)}{(8, 0.6)}{node[right] {E}}
    \ionode{(poutdown)}{(8, -0.6)}{node[right] {F}}
    \mixednode{(pmid)}{(6.5, 0)}{node[right] {C}}
    \mixednode{(pwayup)}{(6.5, 0.6)}{node[above right] {B}}
    \mixednode{(pmap1)}{(5, 1.2)}{node [left] {A'}}
    \mixednode{(pmap2)}{(6.5, 1.8)}{node [right] {B'}}
    \mixednode{(pwaydown)}{(6.5, -0.6)}{node[below] {D}}

    \choice{(pin)}{(pmap1)}{node [left] {$dist_0$}}
    \filter{(pmap1)}{(pmap2)}{node [above left] {x=1?}}
    \syncdrain{(pmap2)}{(pwayup)}{}

    \lossysync{(pin)}{(pwayup)}{}
    \lossysync{(pin)}{(pwaydown)}{}
    \syncdrain{(pin)}{(pmid)}{}
    \sync{(pwayup)}{(pmid)}{}
    \sync{(pwaydown)}{(pmid)}{}

    \sync{(pwayup)}{(poutup)}{}
    \sync{(pwaydown)}{(poutdown)}{}
\end{tikzpicture}

    \caption{From Router to Probabilistic Router}
    \label{fig:router}
\end{figure}

We attach a new path $A\rightarrow A'\rightarrow B'\rightarrow B$ to the original \emph{Router}, including a \emph{StochasticChoice}, a \emph{Filter} and a \emph{SyncDrain}, as depicted in Fig.~\ref{fig:router} (b).
When the \emph{StochasticChoice} channel is triggered, numeric value $0$ or $1$ will be generated, and in turn passed to the \emph{Filter} channel. If the value is $1$, it will be sent to the \emph{SyncDrain} channel $BB'$. In this case, the incoming value has to go through the path $A\rightarrow B\rightarrow E$. Otherwise, if the sampled value is $0$, it will be dropped by the \emph{Filter}, the incoming value will be sent to $F$ as $B$ cannot accept any data from $A$.

% which ensures that the outputs randomly choose two sink ends accordinging to the distrubution, i.e. with the same probability.
% Generally speaking, the random router keeps the same structure with its original version, i.e. one source end and two sink ends, but
% the probabilistic behavior is in consideration.
% To alter the probability of two ends receiving data items, only the parameter of the $StochasticChoice$ channel need update.
% In this way, the nondeterministic router can be turned into a probabilistic router with controllable probability and distribution. 

The corresponding $\nSTA$ of a \emph{Probabilistic Router} can be deduced on the basis of primitive channels' semantics and product operators.
There are two locations in the product $\nSTA$, since the primitive channels in the connector are all synchronous (including only one location) except the \emph{StochasticChoice} channel (having two locations). According to the \emph{product} operator, the locations should be labelled as tuples like \emph{(S0,...,Init,...)}. Here for simplicity we use \emph{Init} and \emph{Ready} instead.
The final result $\nSTA$, after \emph{hiding} all the internal nodes except \emph{A,E,F}, is shown in Fig.~\ref{fig:probabilistic_router}. 
\begin{figure}[H]
\centering
\begin{tikzpicture}
    % State: ACK with different content
    \node[state, double,   	% layout (defined above)
    text width=1.5cm, 	% max text width
    anchor=center] (init) 	% posistion relative to the center of the 'box'
    {\textbf{Init}};

    \node[state,    	% layout (defined above)
    text width=1.5cm, 	% max text width
    right of=init,
    node distance=6cm,
    anchor=center] (ready) 	% posistion relative to the center of the 'box'
    {\textbf{Ready}};

    % draw the paths and and print some Text below/above the graph
    \path
        (init)   edge[->]
            node[below] {[\textbf{i}] \emph{buf := }$\operatorname{B}$\emph{(1,0.5)}}
        (ready)

        (ready)  edge[bend left=30, ->]
            node[below] {[\textbf{A,F}, \emph{buf = 0}] \emph{dF := dA}}
        (init)

        (ready)  edge[bend right=30, ->]
            node[above] {[\textbf{A,E}, \emph{buf = 1}] \emph{dE := dA}}
        (init);
\end{tikzpicture}

\caption{$\nSTA$ of Probabilistic Router}
\label{fig:probabilistic_router}
\end{figure}
\end{example}

% \begin{example}[Embedded Control System]
% A common embedded control system usually comprises a set of sensors to obtain information from the environment, a set of actuators to operate on the environment, and a main processor to process information and give instructions. In the following we show how to use Reo connectors to formalize the coordination part of an embedded control system, which is a simplified version of the embedded controller model with modular redundancy in \cite{Kwia2007}.
% %\ly{
% %    \cite{Kwia2007} proposed an embedded controller model with modular redundancy.  In this paper, we simplify the original model (otherwise the connector would be too complicated to present), and show how to formalize its coordination part as a Reo connector under our framework.
% %}

% \begin{figure}[H]
%     \centering
%     \resizebox{.8\textwidth}{!}{
%         \begin{tikzpicture}
    \mixednode{(i1)}{(1, 1)}{node[left] {$Input_1$}}
    \mixednode{(i2)}{(1, 0)}{node[left] {$Input_2$}}

    \mixednode{(a)}{(3, 1)}{node[above left] {A}}
    \mixednode{(b)}{(3, 0)}{node[above left] {B}}

    \mixednode{(c)}{(4, 0.5)}{node[above ] {C}}
    \mixednode{(f)}{(4, -1.)}{node[below] {$Timeout$}}

    \mixednode{(d)}{(7, 00.5)}{node[above] {D}}
    \mixednode{(e)}{(7, -.25)}{node[below] {E}}

    \mixednode{(o)}{(8, 0.5)}{node[right] {$Output$}}

    \ionode{(m1)}{(5, 0.5)}{node[above] {$M_{in}$}}
    \ionode{(m2)}{(6, 0.5)}{node[above] {$M_{out}$}}

    \blackbox{(i1)}{(a)}{PF}{}
    \blackbox{(i2)}{(b)}{PF}{}

    \lossysync{(a)}{(c)}{}
    \lossysync{(b)}{(c)}{}

    \sync{(c)}{(m1)}{}
    \sync{(m2)}{(d)}{}

    \vrtimer{(c)}{(e)}{(f)}{RT}{}

    \sync{(d)}{(e)}{}
    \sync{(d)}{(o)}{}

    \component{(1,-1)}{(8,2)}{}

\end{tikzpicture}

%     }
%     \caption{Embedded Control System}
%     \label{fig:em}
% \end{figure}

% Sensors, actuators and the main processor are all regarded as components in this example.
% The assumption is, all the components are reliable but their communication is not. Such behavior is captured by a \emph{Probabilistic Filter} connector \textbf{PF} which is defined in Fig.~\ref{fig:em2}.
% The main processor, connected to ports \emph{$M_{in}$} and \emph{$M_{out}$},
% reads data coming from sensors, and passes instructions to the actuator
% through port \emph{Input} and \emph{Output}, respectively.
% The system fails to obtain an input only when both \emph{Probabilistic Filter}s fail.

% \begin{figure}[H]
%     \centering
%     \resizebox{.8\textwidth}{!}{
%         \begin{tikzpicture}
    %probabilistic filter
    \mixednode{(in1)}{(-7, 0)}{node[left] {In}}
    \mixednode{(a1)}{(-6, 1.5)}{node[above] {A}}
    \mixednode{(b1)}{(-4, 1.5)}{node[above] {B}}
    \mixednode{(c1)}{(-4, 0)}{node[below] {C}}
    \mixednode{(out1)}{(-3, 0)}{node[right] {Out}}

    %reset timer
    \mixednode{(in2)}{(-1, 0)}{node[left] {In}}
    \mixednode{(a2)}{(1, 1)}{node[left] {A}}
    \mixednode{(r2)}{(1, 2.2)}{node[above] {Reset}}
    \mixednode{(out2)}{(3, 0)}{node[right] {Timeout}}

    \choice{(in1)}{(a1)}{node [below right] {$x=\mathrm B(1, p)$}}
    \lossysync{(in1)}{(c1)}{}
    \sync{(c1)}{(out1)}{}
    \syncdrain{(b1)}{(c1)}{}
    \filter{(a1)}{(b1)}{node [above=3] {$x=1?$}}

    \choice{(r2)}{(a2)}{node [above right, xshift=0.1cm] {$x=t_{max}$}}
    \ptimer{(in2)}{(a2)}{(out2)}{}

    \component{(-7, -0.6)}{(-3, 2.2)}{}
    \component{(-1, -0.6)}{(3, 2.2)}{}

\end{tikzpicture}

%     }
%     \caption{Probablistic Filter (left) and Reset Timer (right)}
%     \label{fig:em2}
% \end{figure}

% In general, the main processor waits for potential input. However, due to the unreliable channels, we have to set a timeout mechanism to report system failures. In this model, a complete cycle (including data acquisition, data processing and instruction transmission) should be finished within a certain duration. Otherwise a \emph{timeout} signal will be generated by a \emph{Reset Timer} connector \textbf{RT}.

% A \emph{Probabilistic Filter} drops data with a certain probability, i.e. $1 - p$ in this example; while
% the \emph{Reset Timer} with time bound $t_{max}$ is a timer that supports reset operation (triggered by an extra source end \emph{Reset}). Reseting a \emph{Reset Timer} will prevent it from generating \emph{timeout} signals until it receives new value. The formal definition of \emph{Probabilistic Filter} and \emph{Reset Timer} is provided in Fig. \ref{fig:em2}. This also shows how Reo connectors are encapsulated and reused.

% There are 8 locations in the $\nSTA$ of the embedded system,
% consisting of triples which symbolizes the configuration of
%     two probabilistic filters and the reset timer.
% The corresponding $\nSTA$ represented in JANI format\cite{JaniSpec} is provided in Appendix.
% % elements are ommited except locations and edges.
% \end{example}

% \begin{example}[Delayed Channel]
% In this example, we propose an extension of delayed FIFO channels. It is obtained by replacing the timer channel with $StochasticTimer$ channel. The $SyncDrain$ channel serves to assure that the data item written at the source end can be taken exactly after a period of time randomly set by the $StochasticTimer$ channel. However, we can change the parameter of $StochasticTimer$ channel to meet some requirements. For example, if we want to delay the output by one time unit with greater probability or two units, we can set the $dist$ with $0.9\mapsto 1$ and $0.1\mapsto 2$. To a certain extent, it also reflects that a longer time delay interval is unbearable.
% \begin{figure}[H]
%     \centering
%     \begin{tikzpicture}
    % part 1. router
    \ionode{(in)}{(0, 0)}{node[left] {A}} 
    \mixednode{(wp1)}{(1, 0)}{node[below] {B}}
    \mixednode{(wp2)}{(3, 0)}{node[below] {D}}
    \mixednode{(wpup)}{(3, 1)}{node[right] {C}}
    \ionode{(out)}{(4, 0)}{node[right] {E}}

    \sync{(in)}{(wp1)}{}
    \fifoe{(wp1)}{(wp2)}{}
    \Lchannel{timer}{(wp1)}{(wpup)}{1}{v}{+}{node[above left] {\scriptsize dist(params)}}
    \syncdrain{(wpup)}{(wp2)}{}
    \sync{(wp2)}{(out)}{}
    

\end{tikzpicture}
%     \caption{Delayed Channel without Timeout}
% \end{figure}
% Urgency has to be considered in the aforementioned semantics, otherwise an inappropriate case will arise. The data item may be kept in the buffer and the timeout signal is just dropped in the meanwhile. If we don't consider urgency, the semantics which is exactly in the form of $\nSTA$ can be simply drawn as the following figure.

% \begin{figure}[H]
%     \centering
%     \begin{tikzpicture}
    % State: ACK with different content
    \node[state,    	% layout (defined above)
    text width=2.5cm, 	% max text width    
    anchor=center] (emptyinit) 	% posistion relative to the center of the 'box'
    {Empty, Init};
    
    \node[state,    	% layout (defined above)
    text width=2.5cm, 	% max text width
    right of=emptyinit,
    node distance=5.5cm,
    anchor=center] (emptypending) 	% posistion relative to the center of the 'box'
    {Empty, Pending};

    \node[state,    	% layout (defined above)
    text width=2.5cm, 	% max text width
    above of=emptyinit,
    node distance=3.5cm,
    anchor=center] (fullinit) 	% posistion relative to the center of the 'box'
    {Full, Init};

    \node[state,    	% layout (defined above)
    text width=2.5cm, 	% max text width
    right of=fullinit,
    node distance=5.5cm,
    anchor=center] (fullpending) 	% posistion relative to the center of the 'box'
    {Full, Pending};
    
    % draw the paths and and print some Text below/above the graph
    \path
        (emptyinit)   edge[bend left=20, ->] 
            node[below left] {
                \begin{tabular}{c}
                    [\textbf{A,B}] \emph{bd.buf := dA}, \\
                    \emph{bc.delay := dist(params)}
                \end{tabular}
            }
        (fullpending)

        (fullpending)   edge[bend left=20, ->] 
            node[right] {
                \begin{tabular}{c}
                    [\textbf{C,D,E}, \emph{bc.t = bc.delay}] \\
                    \emph{dE := bd.buf},
                \end{tabular}
            }
        (emptyinit)

        (fullpending) edge[->]
            node[above] {[\textbf{i}, $bc.t > bc.delay$]}
        (fullinit)
        ;
\end{tikzpicture}
%     \caption{Semantics of Delayed Channel without Timeout}
% \end{figure}
% \end{example}

\begin{example}[Expiring Timer]
    \label{exp:exptimer}
    
    In Timed Reo \cite{Meng2012}, different types of timer channels are proposed to capture real-time behaviors in different practical scenarios, including: \emph{OffTimer} that allows the timer to terminate the counting process when a certain signal is received, \emph{RSTTimer} that allows the timer to reset and restart its counting process, and \emph{EXPTimer} that makes the timer produce the TIMEOUT signal immediately when a certain signal is received.

    
    In this paper we take \emph{EXPTimer} as an example to show how \emph{pTimer} is used to encode the previous timers. In this example, the default delay time is denoted by $t$. (See Fig.~\ref{fig:exptimer})
    \begin{figure}
        \centering
        \resizebox{0.6\textwidth}{!}{
            \begin{tikzpicture}[scale=1.4]

\tikzstyle{every node}=[font=\small]
\tikzstyle{label}=[draw=none]

\ionode{(A)}{(-0.3, 0)}{node [left=2] {A}}
\mixednode{(B)}{(1,1)}{node [above=2] {B}}
\mixednode{(C)}{(1,-1)}{node [below=2] {C}}
\mixednode{(D)}{(4,1)}{node [above=2] {D}}
\xrouter{(E)}{(3,0)}{node [above=2] {E}}
\mixednode{(F)}{(2,-1)}{node [below=2] {F}}
\mixednode{(G)}{(4,-1)}{node [below=2] {G}}
\ionode{(H)}{(5.3, 0)}{node [right=2] {H}}

\filter{(A)}{(B)}{node [xshift=-2,yshift=2,sloped,anchor=south,auto=false] {\small{$dA\neq \mbox{EXP}$}}}
\filter{(A)}{(C)}{node [xshift=-2,yshift=-2,sloped,anchor=north,auto=false] {\small{$dA=\mbox{EXP}$}}}
\fifoe{(B)}{(D)}{}
\sync{(F)}{(G)}{}
\map{(C)}{(F)}{node [above=2] {$dF:=t$}}
\map{(G)}{(H)}{node [xshift=-2,yshift=-2,sloped,anchor=north,auto=false] {\scriptsize{\emph{dH:=TIMEOUT}}}}
\sync{(E)}{(H)}{}

\syncdrain{(D)}{(E)}{}
\syncdrain{(D)}{(G)}{}

\draw (B) -- (1,0);
\ptimer{(1,0)}{(F)}{(E)}{node [above] {$t$}}

\end{tikzpicture}
        }
        \caption{Encode \emph{EXPTimer} with \emph{pTimer}}
        \label{fig:exptimer}
    \end{figure}
    
    Basically, this connector divide the incoming values into two classes: expiring signals and normal values. For a normal value, it goes into the \emph{pTimer} channel, and its copy is temporarily stored in the \emph{FIFO1} channel to show that the \emph{pTimer} is activated. When the counting process finished successfully without interruption, the buffered value will be dropped due to the \emph{SyncDrain} channel, and a TIMEOUT signal will be sent to $H$.

    On the other hand, when an expiring signal is caught, it will be replaced by the default delay value $t$ and sent to the $T$ end of the \emph{pTimer}. According to the semantics, the \emph{pTimer} will be reset immediately, and the buffered value will be dropped while sending out the TIMEOUT signal.
    
\end{example}
